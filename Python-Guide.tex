\documentclass[a4paper,9pt,twocolumn]{ctexart}
\usepackage[a4paper,left=1cm,right=1cm,top=1.5cm,bottom=1.5cm]{geometry}
\usepackage{amsmath,amssymb}
\usepackage{minted}
\usepackage{xcolor}
\usepackage{hyperref}

\setlength{\parindent}{0pt}
\setlength{\columnsep}{1.5em}
\pagestyle{plain}

\title{Python在算法竞赛中的使用指南}
\author{scutsky}
\date{\today}

\begin{document}
\maketitle

\subsection{Python 在算法竞赛中的使用}

\subsubsection{Python 优势}
\begin{itemize}
\item 语法简洁,开发效率高
\item 内置高精度整数,无需担心溢出
\item 丰富的内置数据结构(list, dict, set, tuple)
\item 强大的内置函数和标准库
\item 适合处理字符串、大数运算等问题
\end{itemize}

\subsubsection{常用技巧}

\textbf{快速输入输出详解:}

\textit{1. 基础输入输出优化}
\begin{minted}{python}
import sys

# 最重要的优化:替换input函数
input = sys.stdin.readline
# 原理:input()每次调用都会刷新缓冲区,而readline()批量读取

# 读取单个整数(注意strip()去除换行符)
n = int(input().strip())
# 或者更简洁(int会自动处理空白符)
n = int(input())

# 读取多个整数
a, b, c = map(int, input().split())

# 读取整数列表
arr = list(map(int, input().split()))

# 读取字符串(需要strip去除换行符)
s = input().strip()

# 快速输出(减少缓冲区刷新)
print(result, flush=False)  # 默认False,不立即刷新
\end{minted}

\textit{2. 大量数据的极速输入}
\begin{minted}{python}
import sys
from io import StringIO

# 方法1:一次性读取所有输入
data = sys.stdin.read().split()
index = 0

def next_int():
    global index
    result = int(data[index])
    index += 1
    return result

def next_str():
    global index
    result = data[index]
    index += 1
    return result

# 使用方法
n = next_int()
m = next_int()
arr = [next_int() for _ in range(n)]

# 方法2:使用缓冲区读取
def fast_input():
    buffer = []
    while True:
        line = sys.stdin.readline()
        if not line:
            break
        buffer.extend(line.split())
    return iter(buffer)

# 方法3:预读取固定行数
def read_lines(n):
    return [sys.stdin.readline().strip() for _ in range(n)]
\end{minted}

\textit{3. 批量输出优化}
\begin{minted}{python}
import sys

# 方法1:收集结果后一次性输出
results = []
for i in range(n):
    # 计算结果
    result = process(i)
    results.append(str(result))

# 一次性输出所有结果
print('\n'.join(results))

# 方法2:使用sys.stdout.write(更快但需要手动添加换行)
for i in range(n):
    result = process(i)
    sys.stdout.write(str(result) + '\n')

# 方法3:缓冲区写入
output_buffer = []
for i in range(n):
    result = process(i)
    output_buffer.append(str(result))
    
    # 每1000行输出一次,避免内存问题
    if len(output_buffer) >= 1000:
        sys.stdout.write('\n'.join(output_buffer) + '\n')
        output_buffer.clear()

# 输出剩余部分
if output_buffer:
    sys.stdout.write('\n'.join(output_buffer) + '\n')
\end{minted}

\textit{4. 不同数据类型的快速读取}
\begin{minted}{python}
# 读取二维数组
n, m = map(int, input().split())
matrix = []
for _ in range(n):
    row = list(map(int, input().split()))
    matrix.append(row)

# 一行式读取二维数组
matrix = [list(map(int, input().split())) for _ in range(n)]

# 读取字符数组
s = input().strip()
char_array = list(s)  # 转换为字符列表

# 读取多组测试数据
while True:
    try:
        line = input().strip()
        if not line:  # 空行结束
            break
        # 处理输入
        process(line)
    except EOFError:  # 文件结束
        break

# 读取不定数量的整数
try:
    while True:
        numbers = list(map(int, input().split()))
        if not numbers:
            break
        process(numbers)
except EOFError:
    pass

# 读取直到特定值
while True:
    n = int(input())
    if n == 0:  # 0作为结束标志
        break
    # 处理数据
\end{minted}

\textit{5. 格式化输出技巧}
\begin{minted}{python}
# 控制小数位数
pi = 3.14159
print(f"{pi:.2f}")        # 3.14
print("{:.3f}".format(pi)) # 3.142

# 科学计数法
large_num = 1234567890
print(f"{large_num:.2e}")  # 1.23e+09

# 左右对齐和填充
name = "Alice"
print(f"{name:>10}")       # "     Alice" 右对齐
print(f"{name:<10}")       # "Alice     " 左对齐  
print(f"{name:^10}")       # "  Alice   " 居中
print(f"{name:*^10}")      # "**Alice***" 用*填充居中

# 进制转换输出
num = 255
print(f"{num:b}")          # 11111111 二进制
print(f"{num:o}")          # 377 八进制
print(f"{num:x}")          # ff 十六进制
print(f"{num:X}")          # FF 大写十六进制

# 数字千位分隔符
big_num = 1234567
print(f"{big_num:,}")      # 1,234,567
\end{minted}

\textit{6. 性能对比和选择建议}
\begin{minted}{python}
# 速度对比(从快到慢):
# 1. sys.stdin.read() + split()           最快,适合超大数据
# 2. sys.stdin.readline()                 次快,通用推荐
# 3. input()                             慢,小数据可用
# 4. sys.stdin.readlines()               慢,内存消耗大

# 输出速度对比(从快到慢):
# 1. sys.stdout.write()                   最快
# 2. print() with 收集后批量输出           次快
# 3. print() 逐行输出                     慢

# 推荐的通用模板
import sys
input = sys.stdin.readline

def main():
    # 读取数据
    n = int(input())
    
    # 处理和输出
    results = []
    for i in range(n):
        # 计算逻辑
        result = solve(i)
        results.append(str(result))
    
    # 批量输出
    print('\n'.join(results))

if __name__ == "__main__":
    main()
\end{minted}

\textbf{常用内置函数:}
\begin{minted}{python}
# 数学函数
import math
math.gcd(a, b)  # 最大公约数
math.lcm(a, b)  # 最小公倍数(Python 3.9+)
math.isqrt(n)   # 整数开方
math.factorial(n)  # 阶乘

# 排序和查找
arr.sort()  # 原地排序
sorted(arr, reverse=True)  # 返回新排序列表
bisect.bisect_left(arr, x)  # 二分查找

# 字符串操作
s.count(sub)    # 统计子串出现次数
s.replace(old, new)  # 替换
''.join(arr)    # 连接字符串列表
\end{minted}

\textbf{数据结构使用:}
\begin{minted}{python}
from collections import deque, Counter, defaultdict
from heapq import heappush, heappop

# 双端队列
q = deque()
q.appendleft(x)
q.popleft()

# 优先队列(最小堆)
heap = []
heappush(heap, x)
min_val = heappop(heap)

# 计数器
cnt = Counter(arr)

# 默认字典
graph = defaultdict(list)
\end{minted}

\subsubsection{注意事项}
\begin{itemize}
\item \textbf{时间复杂度:} Python 比 C++ 慢 10-100 倍,选择题目需谨慎
\item \textbf{递归限制:} 默认递归深度限制 1000,需要时使用 \texttt{sys.setrecursionlimit()}
\item \textbf{浮点精度:} 使用 \texttt{decimal} 模块处理高精度小数
\item \textbf{大数运算:} Python 原生支持任意精度整数
\item \textbf{列表复制:} 注意浅拷贝和深拷贝的区别
\end{itemize}

\textbf{递归深度设置:}
\begin{minted}{python}
import sys
sys.setrecursionlimit(10**6)
\end{minted}

\textbf{高精度小数:}
\begin{minted}{python}
from decimal import Decimal, getcontext
getcontext().prec = 50  # 设置精度
result = Decimal('1') / Decimal('3')
\end{minted}

\subsubsection{适用场景}
\begin{itemize}
\item 大数运算、高精度计算
\item 字符串处理密集的问题
\item 图论中需要复杂数据结构的问题
\item 动态规划(状态较少时)
\item 贪心算法
\item 简单的数学问题
\end{itemize}

\subsubsection{不适用场景}
\begin{itemize}
\item 对时间要求极严格的题目
\item 需要大量循环的数值计算
\item 复杂的计算几何
\item 高频率的数组操作
\end{itemize}

\subsubsection{常用标准库详解}

\textbf{itertools模块:}
\begin{minted}{python}
import itertools

# 排列组合
list(itertools.permutations([1,2,3], 2))     # 排列 A(3,2)
list(itertools.combinations([1,2,3], 2))     # 组合 C(3,2)
list(itertools.combinations_with_replacement([1,2], 3))  # 有重复组合

# 笛卡尔积
list(itertools.product([1,2], [3,4]))       # [(1,3),(1,4),(2,3),(2,4)]

# 累积运算
list(itertools.accumulate([1,2,3,4]))       # [1,3,6,10] 前缀和
list(itertools.accumulate([1,2,3,4], max))  # [1,2,3,4] 前缀最大值

# 分组工具
for k, g in itertools.groupby([1,1,2,2,2,3]):
    print(k, list(g))  # 按连续相同元素分组

# 无限迭代器
itertools.count(10, 2)    # 10, 12, 14, 16, ...
itertools.cycle([1,2,3])  # 1, 2, 3, 1, 2, 3, ...
itertools.repeat(5, 3)    # 5, 5, 5
\end{minted}

\textbf{bisect模块(二分查找):}
\begin{minted}{python}
import bisect

arr = [1, 3, 5, 7, 9]
# 查找插入位置
pos = bisect.bisect_left(arr, 6)    # 3,第一个>=6的位置
pos = bisect.bisect_right(arr, 5)   # 3,第一个>5的位置

# 插入元素并保持有序
bisect.insort(arr, 6)  # arr变为[1,3,5,6,7,9]

# 应用:查找范围内元素个数
def count_range(arr, left, right):
    return bisect.bisect_right(arr, right) - bisect.bisect_left(arr, left)
\end{minted}

\textbf{functools模块:}
\begin{minted}{python}
import functools

# 缓存装饰器(记忆化)
@functools.lru_cache(maxsize=None)
def fibonacci(n):
    if n <= 1: return n
    return fibonacci(n-1) + fibonacci(n-2)

# reduce函数
from functools import reduce
result = reduce(lambda x, y: x + y, [1,2,3,4])  # 10

# 偏函数应用
multiply = lambda x, y: x * y
double = functools.partial(multiply, 2)
print(double(5))  # 10
\end{minted}

\textbf{字符串高级操作:}
\begin{minted}{python}
import string

# 字符常量
string.ascii_lowercase    # 'abcdefghijklmnopqrstuvwxyz'
string.ascii_uppercase    # 'ABCDEFGHIJKLMNOPQRSTUVWXYZ'
string.digits            # '0123456789'

# 字符串方法
s = "hello world"
s.startswith("he")       # True
s.endswith("ld")         # True
s.find("wo")            # 6,找不到返回-1
s.rfind("l")            # 9,从右开始找
s.split()               # ['hello', 'world']
s.isdigit()             # False,是否全为数字
s.isalpha()             # False,是否全为字母
s.zfill(10)             # '00000hello',左补零
\end{minted}

\textbf{位运算技巧:}
\begin{minted}{python}
# 常用位运算
n & (n-1)        # 清除最低位的1
n & -n           # 获取最低位的1(lowbit)
bin(n).count('1') # 统计1的个数
n.bit_length()    # 二进制长度

# 快速幂
def pow_mod(base, exp, mod):
    result = 1
    while exp > 0:
        if exp & 1:
            result = (result * base) % mod
        base = (base * base) % mod
        exp >>= 1
    return result
\end{minted}

\textbf{图论常用模板:}
\begin{minted}{python}
from collections import deque, defaultdict

# BFS模板
def bfs(graph, start):
    visited = set([start])
    queue = deque([start])
    
    while queue:
        node = queue.popleft()
        for neighbor in graph[node]:
            if neighbor not in visited:
                visited.add(neighbor)
                queue.append(neighbor)

# DFS模板
def dfs(graph, node, visited=None):
    if visited is None:
        visited = set()
    visited.add(node)
    
    for neighbor in graph[node]:
        if neighbor not in visited:
            dfs(graph, neighbor, visited)

# Dijkstra最短路径
import heapq
def dijkstra(graph, start):
    dist = defaultdict(lambda: float('inf'))
    dist[start] = 0
    pq = [(0, start)]
    
    while pq:
        d, u = heapq.heappop(pq)
        if d > dist[u]:
            continue
        for v, w in graph[u]:
            if dist[u] + w < dist[v]:
                dist[v] = dist[u] + w
                heapq.heappush(pq, (dist[v], v))
    return dist
\end{minted}

\textbf{动态规划常用技巧:}
\begin{minted}{python}
# 一维DP
dp = [0] * (n + 1)
for i in range(1, n + 1):
    dp[i] = max(dp[i-1], dp[i-2] + arr[i])

# 二维DP(滚动数组优化)
prev = [0] * (m + 1)
curr = [0] * (m + 1)
for i in range(1, n + 1):
    for j in range(1, m + 1):
        curr[j] = prev[j] + grid[i][j]
    prev, curr = curr, prev

# 记忆化搜索
from functools import lru_cache
@lru_cache(maxsize=None)
def dp(i, j):
    if i == 0 or j == 0:
        return 1
    return dp(i-1, j) + dp(i, j-1)
\end{minted}

\textbf{数学算法:}
\begin{minted}{python}
# 最大公约数和最小公倍数
import math
def gcd(a, b):
    while b:
        a, b = b, a % b
    return a

def lcm(a, b):
    return a * b // gcd(a, b)

# 素数筛法
def sieve(n):
    is_prime = [True] * (n + 1)
    is_prime[0] = is_prime[1] = False
    
    for i in range(2, int(n**0.5) + 1):
        if is_prime[i]:
            for j in range(i*i, n + 1, i):
                is_prime[j] = False
    
    return [i for i in range(2, n + 1) if is_prime[i]]

# 快速矩阵幂
def matrix_mult(A, B, mod):
    n = len(A)
    C = [[0] * n for _ in range(n)]
    for i in range(n):
        for j in range(n):
            for k in range(n):
                C[i][j] = (C[i][j] + A[i][k] * B[k][j]) % mod
    return C

def matrix_power(mat, exp, mod):
    n = len(mat)
    result = [[1 if i == j else 0 for j in range(n)] for i in range(n)]
    base = [row[:] for row in mat]
    
    while exp > 0:
        if exp & 1:
            result = matrix_mult(result, base, mod)
        base = matrix_mult(base, base, mod)
        exp >>= 1
    return result
\end{minted}

\textbf{常见错误和避免方法:}
\begin{itemize}
\item \textbf{整数除法:} Python3中 \texttt{/} 返回浮点数,使用 \texttt{//} 进行整数除法
\item \textbf{列表初始化:} \texttt{[[0]*m for \_ in range(n)]} 而不是 \texttt{[[0]*m]*n}
\item \textbf{浮点比较:} 使用 \texttt{abs(a-b) < eps} 而不是 \texttt{a == b}
\item \textbf{字符串拼接:} 大量拼接时使用 \texttt{join()} 而不是 \texttt{+=}
\item \textbf{深拷贝:} 使用 \texttt{copy.deepcopy()} 或列表推导式
\end{itemize}

\textbf{性能优化建议:}
\begin{itemize}
\item 尽量使用内置函数和标准库
\item 避免频繁的字符串操作
\item 使用生成器表达式节省内存
\item 合理使用 \texttt{@lru\_cache} 装饰器
\item 关键循环可以考虑用 C++ 重写
\end{itemize}

\end{document}